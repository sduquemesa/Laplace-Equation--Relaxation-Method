\message{ !name(Resumen.tex)}\documentclass[twocolumn,amssymb,amsmath]{revtex4}

\usepackage{graphicx}
\usepackage[spanish]{babel}
\usepackage[utf8]{inputenc}
\usepackage{tabulary}


\begin{document}

\message{ !name(Resumen.tex) !offset(3) }
 & \centerline{\textit{Instituto de Física, Universidad de Antioquia.}}
  \end{tabular}

\end{minipage}


\title{Fuerza Magnética.\\
                \small{Magnetic force.}}                        %titulo
\author{Sebastián Duque Mesa, Rubén Dario Vargas}       %autores
\affiliation{Universidad de Antioquia.}

\begin{abstract}
                                
  \vspace{0.5cm}
  \begin{picture}(1,1)(1,1) \put(1,10){\line(1,0){380}} \end{picture}                   
  \begin{center}\textbf{Resumen.}\end{center}

  %---------------------------------------------------------
  En esta practica experimental se comprueba la validez de la ley de Biot-Savart y la ley de Inducción de Faraday.
  %---------------------------------------------------------
 
  {\footnotesize Palabras Claves: Campo magnético, ley de Faraday, ley de Biot-Savart.} %palabras claves
                                
  \begin{center}\textbf{Abstract.}\end{center}

  %---------------------------------------------------------
  In this lab we want to proof the validity of the Biot-Savart Law and the Faraday Induction Law.
  %---------------------------------------------------------
  
  {\footnotesize Keywords: Magnetic Field, Faraday law, Faraday induction law.} %keywords
                                
  \vspace{0.8cm}
  \begin{picture}(1,1)(1,1) \put(1,10){\line(1,0){380}} \end{picture}
  
\end{abstract}

\vspace{0.5cm}
\maketitle

\[
x^2
\]

\end{document}
\message{ !name(Resumen.tex) !offset(-56) }
